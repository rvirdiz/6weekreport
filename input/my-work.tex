
\subsubsection{Registered Clients Register}
It becomes easy to manage things when placed collectively. 
Priviously there was no facility in software to see all the registered 
clients  on single click, were saved only at the backend Now I have added 
this feature in registered clients register under "Registers" menu of software. 
Clicking on which user can see a list of all registered clients with their 
respective details, can see the previous work of any client, add new work/job from 
there only.
\subsubsection{Formats of reports}
Like medical reports, after the testing of materials like Steel, Cement, Bricks,
Ground Water, Wood and Tiles etc., proper reports are made by software itself.
I made formats of reports for different materials as per the tests performed 
on a particular material.
\subsubsection{Phonetic Search}
Implementing Phonetic Search in the software was the actual work to be done.
Whenever there is a mob there is a need to find, a need to search. 
A search is the organized pursuit of information. Somewhere in a collection
of documents, Web pages, and other sources, there is information that you want
to find, but you have no idea where it is. 

Ever since the world wide web became the engine of our lives, search has been 
the holy grail for developers and companies. Search is essential, search is 
important, we keep searching the whole day, some say even our life is nothing
but a search itself. 

To search anything from WWW, we
needed search engines. A web Search engine is a softwares  that is desgned 
to search information from World Wide Web i.e they are the websites that 
fetch information from other websites. All search engines use algorithms to 
provide the required results. The data they search for is indexed and is 
easy to find for user, but its work flow is much complicated.
\begin{figure}[h]
{\bf History:} 
Beginning with Archie in 1990, considered the first search engine, moving
on to Excite and Lycos and Infoseek, by the mid 90s there was a veritable 
flood of search engines, particularly after Google showed how it should be 
done in 1996. The complexity of the algorithms was now matched only by the
 voracious appetite of searchers as the number of pages to be indexed ran 
into billions. Invariably, a lot of them positioned themselves as specialized
engines—for kids or jobs or tech or entertainment. As the original super spider
, AltaVista, shuts down, here’s a brief history of some of the better known
search engines through the years:
\end{figure}
\begin{itemize}
\item 1990: Archie—the very first search engine
\item 1991: Veronica and Jughead
\item 1992: Vlib
\item 1993: Excite and World wide web wanderer
\item 1994: AltaVista, Galaxy, Yahoosearch, Infoseek, Webcrawler, Lycos
\item 1995: Looksmart
\item 1996: Google, HotBot, Inktomi
\item 1997: Ask.com
\item 1998: MSN; dmoz
\item 1999: Alltheweb
\item 2006: Microsoft Livesearch
\item 2009: Microsoft Bing
\end{itemize}


{\bf Role in TCC Automation Software:}
Software has a huge database, and a pain to find something if there's no search 
capability. I was given the task to relieve this pain. Searching capability and with
phonetic search, makes it easy to find the resembling spellings too.

{\bf Previous search:}
Previously there was a search module in this software 
wich was a live search. There were django queries that fetched the entered
keywords from database. A django function was used there. The results of that search were 
unexpected. User had to enter a space before the keywords. It didn't even search for all 
the matches. It was totally inefficient and staff members also wanted something very good that
could make their work easy.

%\subsubsection{Additional note}
%There was static note section in performa bill which contained some
%instructions for client. These instructions include the following:\\
%\begin{itemize}
%\item Payment may be made via D.D./Cash/Online in favour of Director,
%Guru Nanak Dev Engg. College, Ludhiana.
%\item The Testing/Work shall be done on Receipt of this payment.
%\item Three Labourers for soil testing work to be provided by the 
%Department/Client at each site.
%\end{itemize}

\subsubsection{Implementation of phonetic search}
To implement phonetic search in the software, a soundex algorithm is used.
Soundex is a phonetic algorithm for indexing names by sound, as pronounced 
in English. The Soundex heuristic can be used for identifying names that 
sound alike but are spelled differently. The Soundex algorithm is a standard 
feature of MS SQL and Oracle database management systems to search for similar 
sounding names. 
\begin{figure}[h]
\centering \includegraphics[scale=0.60]{images/table.png}
\caption{Soundex codes}
\end{figure}

Here if we are searching for the name “Smith” or “Smythe”, while the words are
 different, the pronunciation is the same. The Soundex signature for both is the 
same “S530”.

\begin{itemize}
\item Following are the steps to implement the algorithm: 
\item Ignore all characters in the string being encoded except for the English letters,
 A to Z.
\item The first letter of the Soundex code is the first letter of the string being encoded.
\item After the first letter in the string, do not encode vowels or the letters H, W and Y.
\item Assign a numeric digit between one and six to all letters except the first using the 
following mappings:

    1: B, F, P or V

    2: C, G, J, K, Q, S, X, Z

    3: D, T

    4: L

    5: M, N

    6: R

\item Where any adjacent digits are the same, remove all but one of those digits unless a vowel,
 H, W or Y was found between them in the original text.

\item Force the code to be four characters in length by padding with zero characters or by truncation.
\end{itemize}

\subsubsection{Future scope:}

To implement a better search in a software, or a website, one need to set the 
requirements about what kind of searching algos can be implemented and what 
is the requirement. There are various searching techniques and algorithms
which could be used and were studied: 
\begin{enumerate} 
\item Autocomplete
\item Spelling suggestions
\end{enumerate}

\subsubsection{Adding New Calendar}
Looks of a software matter a lot, because it has direct impact 
on user's mind. The calendar previously used was not proper. Month and
year were not even visible properly. Here is the difference.
\begin{figure}[ht]
\centering
\begin{minipage}[b]{0.45\linewidth}
\includegraphics[scale=0.70]{images/calp.jpg}
\end{minipage}
\quad
\begin{minipage}[b]{0.45\linewidth}
\includegraphics[scale=0.64]{images/caln.jpg}
\end{minipage}
\caption{Comparison of previous(left) and new(right) calendar}
\end{figure}
\newpage
\subsubsection{Installation Script}
It becomes very tedious to install any software manualy and then remove 
lot of errors and check missing dependencies. Because its Autiomation 
Software, it's installation needed to be automated. I have created a 
bash script, using which anyone can install it easily. \\
\newpage
\begin{figure}[h]

{\bf Insatallation design:}

\centering \includegraphics[scale=0.43]{images/inst4.png}
\caption{(a)Flow Chart for Installation}
\end{figure}
\newpage
\begin{figure}[h]
\centering \includegraphics[scale=0.40]{images/inst3.png}
\caption{(b)Flow Chart for Installation}
\end{figure}
\newpage
\newpage
\begin{figure}[h]
\centering \includegraphics[scale=0.40]{images/inst2.png}
\caption{(c)Flow Chart for Installation}
\end{figure}
\newpage
\begin{figure}[h]
\centering \includegraphics[scale=0.40]{images/inst1.png}
\caption{(d)Flow Chart for Installation}
\end{figure}
\newpage
