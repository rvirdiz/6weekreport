\section{System Design} : Systems design is the process or art of defining 
the architecture, components, modules, interfaces, and data for a 
system to satisfy specified requirements. One could see it as the 
application of systems theory to product development. There is some 
overlap with the disciplines of systems analysis, systems architecture 
and systems engineering.
\begin{itemize}
\item  External design: External design consists of conceiving, 
planning out and specifying the externally observable characteristics 
of the software product. These characteristics include user displays 
or user interface forms and the report formats, external data sources 
and the functional characteristics, performance requirements etc. 
External design begins during the analysis phase
and continues into the design phase.
\item  Logical design: The logical design of a system pertains to an 
abstract representation of the data flows, inputs and outputs of the 
system. This is often conducted via modeling, which involves a 
simplistic (and sometimes graphical) representation of an actual 
system. In the context of systems design, modeling can undertake the 
following forms, including:
\begin{itemize}
\item Data flow diagrams
\item Entity Relationship Diagrams
\end{itemize}
\item  Physical design: The physical design relates to the actual 
input and output processes of the system. This is laid down in terms 
of how data is input into a system, how it is verified/authenticated, 
how it is processed, and how it is displayed as output.
\end{itemize}
\newpage
\section{Design Notations}
{\bf Data Flow diagrams}:
\image{0.6}{images/sss1.png}{DFD Notation}\\
\newpage
{\bf Flow Charts}:
\image{0.5}{images/sss2.png}{Flow Chart Notations}\\
%Entity Relationship Diagrams:
%\image{0.6}{images/sss3.png}{ER Diagram Notation}\\
{\centering \bf Detailed Design}
We basically describe the functionality of the system internally. 
The internal design describes how data is flowing from database to the 
user and how they both are internally connected. For this reason we 
can show the design of the system in detailed manner by many ways: \\\\
{\bf Flowchart } A flowchart is a type of diagram that represents an
algorithm or process, showing the steps as boxes of various kinds, and
their order by connecting them with arrows. This diagrammatic
representation can give a step-by-step solution to a given problem.
Process operations are represented in these boxes, and arrows
connecting them represent flow of control. Data flows are not
typically represented in a flowchart, in contrast with data flow
diagrams; rather, they are implied by the sequencing of operations.
Flowcharts are used in analyzing, designing, documenting or managing a
process or program in various fields
\newpage
\section{Database Design}
\image{0.4}{images/design.png}{Database Design}

